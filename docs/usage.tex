\documentclass{article}
\title{Centi network | Usage guide}
\author{Centi Project}

\usepackage{listings}
\usepackage{color}
\usepackage{indentfirst}
\renewcommand\lstlistingname{Quelltext}

\lstset{ % General setup for the package
    language=Bash,
    numbers=left,
    frame=tb,
    tabsize=4,
    columns=fixed,
    showstringspaces=false,
    showtabs=false,
    keepspaces,
    commentstyle=\color{gray},
    keywordstyle=\color{blue}
}


\begin{document}
\maketitle
\newpage
\tableofcontents
\newpage
\section{Overview}
	\subsection{What is Centi?}
	Centi is a some kind of framework for building anonymous
	networks upon controlled communication channels.
	It is made in order to give people in censured countries access
	to anonymous and secure communications even if only dictatorish services will work.

	Centi is not magic, although it is designed to make all of
	the network participants as anonymous as possible in described situation.

	\subsection{How does it work?}
	The core of Centi network is its modules. A module is a part of
	program which allows users to communicate in some way, for example,
	via popular platforms like GitHub, bluetooth devices, WebRTC, etc.
	So in case if one channel of communications will be blocked, an another
	one will be adapted quickly and users will be able to communicate with each other again.
	
	Centi also supposed to use different steganography methods to make
	high latency communications. If you are interested in methods of steganography,
	please, review our steganography module and suggest any improvements if you know
	how to make it better.

	Currently steganography methods for PNG, JPEG, GIF, WAV, MP3, FLAC, ZIP and some
	other files are supported, we are looking to make this list bigger in the future.

\section{Installation}
	Clone repository and build Centi with following commands:
	\begin{lstlisting}
git clone https://github.com/random-username-dummy/centi
cd centi
make release
	\end{lstlisting}
	If you want, you can move binary to any location in your PATH.

\section{Quick start}

	For the first time you run Centi, it asks you for a password.
	This is a password for all of your configuration files, logs, etc.
	You shouldn't use you local root or user password, really, just
	create a new one which is easy for you to remember (but not easy for attacker to guess).
	Every time you run Centi it will ask you for this password so don't lose it.
	
	If you have accidentially lost password, you can just remove ~/.centi directory
	and start the network again. But all of your configuration will be lost and you'll
	spend some time to reabilitate it.

	So, let's continue.
	You can run the network using the following command:
	\begin{lstlisting}
./centi run
	\end{lstlisting}
	
	For the first time network won't connect to any nodes because it is not configured yet.
	In case something went wrong and you didn't manage to connect to network after setting
	up connection parameters, read the logs with following command:
	\begin{lstlisting}
./centi readlog
	\end{lstlisting}
	
	This will show all the errors the program has encounted.

\section{Configuration}
	On first execution the program also will generate default configuration file
	which MUST be edited in order for you to connect the network.
	Edit configuration with following command:
	\begin{lstlisting}
./centi editconf
	\end{lstlisting}
	
	Centi will use Vim as a default text editor. You can change this behaviour by
	adding CENTI\_EDITOR environment variable, for example:
	\begin{lstlisting}
export CENTI\_EDITOR=/usr/bin/nano
	\end{lstlisting}
	
	So, what do we see here?
	\subsection{Network configuration}
	First of all, you can see what configuration is split onto some smaller confguration.
	The first configuration is network configuration. Here you can see options,
	related to network performance: min\_delay, max\_delay, keys\_collection\_delay, queue\_size and packet\_size.
	If you aren't sure, leave them as is for now.

	Next you can see privacy-related settings: accept\_unknown, send\_known\_peers and
	ephemerial\_mode. You can read more about these settings in Advanced options section.

	Most vital parameters in network configuration are network\_key and network\_subkeys.

	network\_key is a key of your subnetwork. Only you and your trusted nodes (friends)
	use it for key exchange. And yes, nothing works without it.

	network\_subkeys is a dictionary which contains peers data in format
	"username": "pre-shared-secret". Username is a username on platform,
	ip address, email address or whatever your modules require.

	Let's look at the example.
	We have Alice and Bob. They have agreed what their PSK is "i-love-freedom" (please, use
	more reliable one).

	Alice adds line '"bob": "i-love-freedom",' in her network subkeys configuration and Bob adds line
	'"alice": "i-love-freedom",'.
	
	After that they are able to exchange public keys securely.

	The last parameter, key\_distribution\_parameters, is used by Centi modules. You can specify
	nothing here. In this case module must use it's default public key distribution algorithms.

	\subsection{Local server configuration}
	This section is related to configuration of your local server. This one is used
	for handling API requests to network. You can pick any port and address you want
	the server to run on.
	
	WARNING: API server does not support authentication yet. Do not allow anyone
	(except yourself) connect to it.

	`pages` parameter contains all the locations of files to use for general HTTP service.
	You can put your applications here, if you want.

	\subsection{Steganography configuration}
	This is the smallest configuration ever. All you need is to specify a folder with decoy files.
	Again, Centi's steganography methods support many files formats, just find a lot of appropriate
	files and put them here.

	\subsection{Logger configuration}
	This is an extra configuration which is not neccessary at all. Leave default if you don't want
	to spend time on this. If you want logs to be encrypted or not encrypted, change `is\_encrypted`
	parameter value and you are good to go.

	\subsection{Platforms data}
	This is a one of the most important configurations. It holds all your modules configuration data (
	api tokens, usernames, repositories names, etc.).

	A single entity of platform data contains the following fields:
	platform - the name of platform.
	args - a dictionary of argument to module.
	channels - a list of channels for I/O. Every channel contains it's name and arguments.
	
	You should find information about correct configuration for concrete module
	in your module's docs.

	\subsection{Database configuration}
	This is even not a section, it's just 3 parameters: database file location, password for
	database (cleartext, because configuration file it's stored in is encrypted anyway)
	and rows limit.

	\subsection{Keys configuration}
	This is an another one vital section. It contains all the pre-shared secrets and your keys.
	Do NOT share your secret(private) key with anyone but give public key to whoever you want to.
	Also do NOT edit your keys by hand.

	\subsection{Conclustion}
	This is all the options you can configure for now. Yeah, there are a lot of them. But these options allow
	you to create your own subnetworks and do whatever you need without changing any single line of code.


\section{Advanced options}
	\subsection{Security settings}
		\paragraph{Ephemerial mode}
		Centi network has such feature as ephemerial mode.
		In this mode every node you are connecting to(even trusted one) does not know any of your public keys.
		This mode is enabled by default but if you want to be authenticated on nodes you trust, no one stops you.
		But be careful.

		\paragraph{Accept unknown connections}
		You can not accept connections from people you don't know. But in this case all your friends
		will have to disable ephemerial mode. Use if you and your friends have bad internet speed and
		want to create your own subnetwork.

		\paragraph{Passwords}
		In order to prevent MitM attacks, Centi uses pre-shared secrets or passwords.
		These passwords are used only for HMAC calculation and authentication.
		PSKs do not directly participate in key exchange or key generation.
		Just agree with your trusted nodes when you will change PSKs and their parameters.
		Also agree on what to do in case if you'll lose your PSKs.
		And please, do this OUTSIDE controlled social media and platforms.
		Use decentralized messengers or, what even better, offline communication channels for this.

	\subsection{Performance settings}
		\paragraph{Buffer size}
		If you want to send more or less data if you do now, just change the buffer size.
		The only requirement for buffer size is what it must be bigger than 2048 bytes (because of public keys sizes).
		Your trusted nodes and other nodes you are connecting to must use the same
		size of buffer so you can use buffer size as some kind of `private network parameter`.
		However, we consider what buffer size should not change network behaviour and
		this should be fixed in future versions.

		\paragraph{Delays between requests}
		Centi network uses randomized delays between sending packages.
		You can change them in configuration but be careful because too small delays tend to hit rate limits.

		\paragraph{Queue size}
		The size of messages queue, this is used only for your messages,
		not for messages which you are resending to other people.
		Any reasonable size is fine.
\section{FAQ}
	There is no actual frequently asked questions so this section is empty for now.
	If you have any questions, contact us on <email-address>.
\end{document}
