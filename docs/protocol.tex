\documentclass{article}
\title{Centi network | Protocol}
\author{Centi Project}

\usepackage{indentfirst}
\usepackage{graphicx}
\usepackage{listings}
\usepackage{color}
\renewcommand\lstlistingname{Quelltext}

\lstset{ % General setup for the package
    language=Go,
    numbers=left,
    frame=tb,
    tabsize=4,
    columns=fixed,
    showstringspaces=false,
    showtabs=false,
    keepspaces,
}

\begin{document}
\maketitle
\newpage
\tableofcontents
\newpage

\section{Introduction}
	Centi is a queue-based network which took main idea from HiddenLake. Briefly speaking,
	it generates a fake package every random period of time and sends it to the network
	in order to anonymize sender. It also sends a package to every network participant
	to anonymize receiver.

\section{General protocol overview}
\section{Key exchange}
	Centi combines both pre-quantum and post-quantum algorithms for key exchange. We use
	x25519 and Kyber512 for key exchange. After that, shared secret is retrieved via HKDF
	with SHA512 hash set. So even if attacker is able to break one of key exchange algorithms
	the final shared secret will stay unknown.

	Centi also uses pre-shared secrets in order to detect and prevent public key substitution
	(MitM attack).

	After successful key exchange, users encrypt their packages using ChaCha20Poly1305 algorithm.
\section{Packages format}
	General package strcture looks like this:
	\begin{lstlisting}
type Packet struct {
	Head	PacketHead		`json:"h"`
	Body	PacketBody		`json:"b"`
}

type PacketHead struct {
	// typ of packet, just 1 digit
	Typ		uint8		`json:"t"`

	// sequence number (to identify, which packets were sent earlier)
	Seq		uint64		`json:"s"`	
	
	// total parts of packet
	Total		uint64		`json:"l"`

	// if the total data in packets is compressed, 0 or 1
	Compressed	uint8		`json:"c"`
}

type PacketBody struct {
	// actual data. very situatively parameter,
	// may be used for both `network` and
	// `application` layers.
	Data		string		`json:"d"`

	// used in order to correctly decode data
	OrigSize	uint		`json:"o"`

	// some extra integrity check
	Hmac		string		`json:"h"`
}
	\end{lstlisting}


\section{Getting new peers info}
	Peer info can be reset inside the network by known peers. If user does not want
	to share their peers, Centi just sends an empty list of peers.

\section{Other security-related features}

\section{Conslusion}
\end{document}
